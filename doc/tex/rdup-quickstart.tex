\documentclass[a4paper, openany]{atroff}
\newcommand{\rdup}{\textbf{rdup}}
\newcommand{\cmd}[1]{\texttt{#1}}
\newcommand{\url}[1]{\texttt{#1}}
\newcommand{\path}[1]{\texttt{#1}}
\newcommand{\flag}[1]{\textit{#1}}
\title{\rdup: Quick Start}
\begin{document}

\chapter*{\rdup: Quick Start}
This is a small guide about setting up and running \rdup{} to make
backups. \rdup{}
supports incremental, compressed, encrypted, local and remote backups and
restores. All these feature are implemented in scripts to
implement an actual backup scheme. The core \rdup{} utility does
nothing but to print out the files names that are to be backed up. A
small set of Perl scripts implements a hardlinked backup strategy.

\begin{atleftbar}
See \url{www.miek.nl/projects/rdup} for more information.
\end{atleftbar}

\section*{Setup}
Get \rdup{} from its website. Configure and install it:
\begin{atalltt}
./configure && make && sudo make install
\end{atalltt}
\rdup{} by default creates a full dump at the first run, after
that it will make incrementals. This can ofcourse be fully tweaked.

\section*{Backup}
The most convenient way to use \rdup{} is to use the wrapper
script \cmd{rdup-simple}. Synopsis:\\
\cmd{rdup-simple [ OPTIONS ] DIR|FILE [ DIR|FILE ... ] DEST} 
\begin{itemize}
\item
Dump \path{/home} to your USB disk on \path{/media/usbdisk}:
\begin{display}
rdup-simple /home /media/usbdisk/\$HOST
\end{display}
\item
Dump compressed \path{/home} to \path{/vol/backup}:
\begin{display}
rdup-simple -z /home /vol/backup/\$HOST
\end{display}
Dump encrypted \path{/home} to \path{/vol/backup} at remote host:
\begin{display}
rdup-simple -k secret_file /home \
    ssh://user@example.nl:/vol/backup/\$HOST
\end{display}
\end{itemize}

\begin{atblock}
For more examples look at rdup-simple(1).
\end{atblock}

\section*{Restore}
The most convenient way to restore with \rdup{} is by using
the script \cmd{rdup-restore}. Synopsis:\\
\cmd{rdup-restore [ OPTIONS ] SOURCE [ SOURCE ... ] DIR}
\begin{itemize}
\item
Restore from the latest backup from May 2006, to \path{/tmp/restore}:
\begin{display}
rdup-restore /vol/backup/elektron/200605/home/miekg /tmp/restore
\end{display}
\item
Restore from the latest remote backup from May 2006. This backup
is encrypted:\\
\begin{display}
rdup-restore -k secret\_file /tmp/restore \
    /vol/backup/elektron/200605/home/miekg /tmp/restore
\end{display}
\end{itemize}

\begin{atblock}
See rdup-restore(1) for more information.
\end{atblock}

\end{document}
